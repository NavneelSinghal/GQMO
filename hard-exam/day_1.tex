\documentclass{qmo}

\qmotime{Time: 5 hours}
\date{May 16th, 2020}
\level{Advanced level}
\qmoday{Day 1}

\begin{document}
\maketitle
\begin{flushright}
\emph{Each problem is worth 7 points.}\nl
\emph{These problems are to be kept confidential till Monday, 18$^\mathit{th}$ May 2020, 1200 hours (GMT).}
\end{flushright}
\begin{problem}
	Let $ABC$ be a triangle with incentre $I$. The incircle of the triangle $ABC$ touches the sides $AC$ and $AB$ at points $E$ and $F$, respectively. Let $\ell_B$ and  $\ell_C$ be the tangents to the circumcircle of $BIC$ at $B$ and $C$, respectively. Show that there is a circle tangent to $EF,$ $\ell_B$ and $\ell_C$ with centre on the line $BC$.
\end{problem}

\begin{problem}
	Geoff has an infinite stock of sweets, which come in $n$ flavours. He arbitrarily distributes some of the sweets amongst $n$ children (a child can get sweets of any subset of all flavours, including the empty set). Call a distribution of sweets $k$\emph{-nice} if every group of $k$ children together has sweets in at least $k$ flavours. Find all subsets $S$ of $\{1,2,\dots,n\}$ such that if a distribution of sweets is $s$-nice for all $s\in S$, then it is $s$-nice for all $s\in \{1,2,\dots,n\}$.
\end{problem}

\begin{problem}
	We call a set of integers \textit{special} if it has $4$ elements and can be partitioned into $2$ disjoint subsets $\{a,b\}$ and $\{c,d\}$ such that $ab-cd=1$. For every positive integer $n$, prove that the set $\{1,2, \ldots , 4n\}$ cannot be partitioned into $n$ disjoint special sets.
\end{problem}

\begin{problem}
	Prove that, for all sufficiently large integers $n$, there exist $n$ numbers $a_1,a_2,\ldots,a_n$ satisfying the following three conditions:
	\begin{itemize}
	\item Each number $a_i$ is equal to either $-1$, $0$ or $1$.
	\item At least $2n/5$ of the numbers  $a_1,a_2,\ldots, a_n$ are non-zero.
	\item The sum $a_1/1+a_2/2+\ldots+a_n/n$ is $0$.
	\end{itemize}
	\emph{Note: Results with $2/5$ replaced by a constant $c$ will be awarded points depending on the value of $c$.}
\end{problem}

\end{document}
