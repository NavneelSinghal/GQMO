\section{Problems}
\subsection*{Problem 1}
Let $ABC$ be a triangle with incentre $I$. The incircle of the triangle $ABC$ touches the sides $AC$ and $AB$ at points $E$ and $F$, respectively. Let $\ell_B$ and  $\ell_C$ be the tangents to the circumcircle of $BIC$ at $B$ and $C$, respectively. Show that there is a circle tangent to $EF,$ $\ell_B$ and $\ell_C$ with centre on the line $BC$.\nl
\textit{Proposed by Navneel Singhal, India}
\subsection*{Problem 2}
Geoff has an infinite stock of sweets, which come in $n$ flavours. He arbitrarily distributes some of the sweets amongst $n$ children (a child can get sweets of any subset of all flavours, including the empty set). Call a distribution of sweets $k$\emph{-nice} if every group of $k$ children together has sweets in at least $k$ flavours. Find all subsets $S$ of $\{1,2,\dots,n\}$ such that if a distribution of sweets is $s$-nice for all $s\in S$, then it is $s$-nice for all $s\in \{1,2,\dots,n\}$.\nl
\textit{Proposed by Kyle Hess, USA}
\subsection*{Problem 3}
We call a set of integers \textit{special} if it has 4 elements and can be partitioned into 2 disjoint subsets $\{a,b\}$ and $\{c,d\}$ such that $ab-cd=1$. For every positive integer $n$, prove that the set $\{1,2, \ldots , 4n\}$ cannot be partitioned into $n$ disjoint special sets.\nl
\textit{Proposed by Mohsen Jamali, Iran}
\subsection*{Problem 4}
Prove that for all sufficiently large integers $n$, there exist $n$ numbers $a_1,a_2,\ldots,a_n$ satisfying the following three conditions:
\begin{enumerate}
\item Each number $a_i$ is equal to either $-1$, $0$ or $1$.
\item At least $2n/5$ of the numbers  $a_1,a_2,\ldots, a_n$ are non-zero.
\item $a_1/1+a_2/2+\ldots+a_n/n=0$.
\end{enumerate}
\textit{Proposed by Navneel Singhal, India, Kyle Hess, USA, and Vincent Jugé, France}
\subsection*{Problem 5}
Let $\mathbb{Q}$ denote the set of rational numbers. Determine all functions $f:
\mathbb{Q}\rightarrow\mathbb{Q}$ such that, for all $x,y \in\mathbb{Q}$,
\[f(x)f(y+1) = f(xf(y))+f(x).\]
\textit{Proposed by Nicolás López Funes and José Luis Narbona Valiente, Spain}
\subsection*{Problem 6}
Decide whether there exist infinitely many triples $(a,b,c)$ of positive integers such that all prime factors of $a!+b!+c!$ are smaller than $2020$.
\nl
\textit{Proposed by Pitchayut Saengrungkongka, Thailand}
\subsection*{Problem 7}
Each integer in ${1, 2, 3, \ldots, 2020}$ is coloured in such a way that, for all positive integers $a$ and $b$ such that $a+b \leq 2020$, the numbers $a$, $b$ and $a+b$ are not coloured with three different colours. Determine the maximum number of colours that can be used. \nl
\textit{Proposed by Massimiliano Foschi, Italy}
\subsection*{Problem 8}
Let $ABC$ be an acute scalene triangle, with the feet of $A, B, C$ onto $BC, CA, AB$ being $D, E, F$ respectively. Suppose $N$ is the nine-point centre of $DEF$, and $W$ is a point inside $ABC$ whose reflections over $BC, CA, AB$ are $W_a, W_b, W_c$ respectively. If $N$ and $I$ are the circumcentre and incentre of $W_aW_bW_c$ respectively, then prove that $WI$ is parallel to the Euler line of $ABC$.\nl
\emph{Note: If XYZ is a triangle with circumcentre O and orthocentre H, then the line OH is called the Euler line of XYZ and the midpoint of OH is called the nine-point centre of XYZ.}\nl
\textit{Proposed by Navneel Singhal, India and Massimiliano Foschi, Italy}
