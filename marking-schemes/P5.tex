\section{Problem 5}
\subsection{Problem}
Let $\mathbb{Q}$ denote the set of rational numbers. Determine all functions $f:
\mathbb{Q}\rightarrow\mathbb{Q}$ such that, for all $x,y \in\mathbb{Q}$,
\[f(x)f(y+1) = f(xf(y))+f(x).\]
\textit{Proposed by Nicolás López Funes and José Luis Narbona Valiente, Spain}

\subsection{Solutions}
We denote the given functional equation by $\mathbf{E}(x,y)$. The functions 
$f: x \mapsto 0$, $f: x \mapsto 2$ and $f: x \mapsto x$ are solutions. We prove below 
that there are no other solutions.

\subsubsection{Solution 1 (Vincent Jugé)}
Let $f$ be a solution other than the two functions $x \mapsto 0$ and $x \mapsto 2$. First,
$\mathbf{E}(0,x-1)$ states that $f(0)(f(x)-2)=0$ for all $x\in\mathbb{Q}$. Since $f$ is not equal to
the function $x \mapsto 2$, it follows that $f(0)=0$.\\\\
Then, since f is not constantly zero, there exists a real number $z$ such that $f(z)\neq 0$. The
equality $\mathbf{E}(z,0)$ states that $f(z)(f(1)-1)=0$, i.e., that $f(1)=1$.\\\\
We prove now, by induction, that $f(n)=n$ for all integers $n \geq 0$. Indeed, this is already the case
for $n = 0$ and $n = 1$ and, if $f(n)=n$ for some integer $n \geq 0$, the equality
$\mathbf{E}(1,n)$ states that
\[f(n + 1) = f(1)f(n + 1) = f(f(n)) + f(1) = n + 1.\]
Furthermore, if $x=\frac{p}{q}$ is a positive rational number, with $p$ and $q$ positive integers,
the equality $\mathbf{E}(x,q)$ proves that $qf(x)=p$, i.e., that $f(x)=x$.\\\\
Now, let us focus on negative rational numbers, and let $a = f(-1)$. The equality $\mathbf{E}(1, -1)$ states
that $f(a) = -1$, and thus that $a < 0$. Then, if $r$ is a negative rational number, the equality
$\mathbf{E}(-1, -r)$ also states that $f(r) = -ar$. It follows that $-1 = f(a) = -a^2$ and, since
$a<0$, that $a=-1$.\\\\
Hence, we conclude that $f(x)=x$ for all $x\in\mathbb{Q}$, which completes the proof.

\subsubsection{Solution 1a (Jhefferson Lopez)}
We demonstrate an alternative way of getting $f(0)=0$ if $f$ is not constantly 2. The equation
$\mathbf{E}(0,-1)$ tells us that $f(0)^2 = 2f(0)$, which means that either $f(0)=0$ or $f(0)=2$. If
$f(0)=2$, then $\mathbf{E}(0,x-1)$ implies that $f(x)=2$ for all $x\in\mathbb{Q}$, which contradicts
our assumption. Therefore $f(0)$ must be 0.

\subsubsection{Solution 2 (Jhefferson Lopez)}
Just like in solution 1 we show that if $f$ is a solution other than $x \mapsto 0$ and 
$x \mapsto 2$, then $f(x) = x$ for all rational numbers $x \geq 0$.\\\\
From the equality $\mathbf{E}(1,-1)$ we get that $f(f(-1))=-1$. Using this, the equality
$\mathbf{E}(1,f(-1))$ tells us that $f(1+f(-1))=1+f(-1)$. The equality $\mathbf{E}(x,f(-1))$ states
that \[f(x)f(1+f(-1)) = f(xf(f(-1)))+f(x)\]
which then, using the two facts we just derived, implies that
\begin{equation}
    f(x)f(-1)=f(-x). \label{f-is-almost-odd}
\end{equation}
Plugging $x=-1$ into (\ref{f-is-almost-odd}), we get that $f(-1)^2=f(1)=1$. This leaves us with the
two cases $f(-1)=1$ or $f(-1)=-1$. If $f(-1)=1$, then we have $f(f(-1))=f(1)=1$, which is a
contradiction to $f(f(-1))=-1$. If $f(-1)=-1$, then (\ref{f-is-almost-odd}) tells us that
$f(x)=-f(-x)$ for all $x\in\mathbb{Q}$. Since we already know $f(x)=x$ for all rational 
$x\geq 0$, this tells us that $f(x)=x$ for all $x\in\mathbb{Q}$, which completes the proof.\\\\
\textit{Remark.} We can also first show $f(n)=n$ for all integers $n\geq
0$, then show $f(x)=-f(-x)$ as mentioned above, and then use $\mathbf{E}(\frac{p}{q},q)$ with
integers $p,q \neq 0$ to conclude that $f(x)=x$ for all $x\in\mathbb{Q}$.

\subsection{Marking scheme}
In every sub-enumeration, the mentioned partials are given, if the mentioned part of the solution
has not been completed. 
\begin{enumerate}
        \marking{1}{Discovering all the solutions (without any additional "solutions" that do not actually
        satisfy the functional equation), and mentioning they all work\footnote{Something along the lines 
        of ``it's obvious that they are solutions" is enough.}}
        \marking{6}{Proving that there can not be any other solutions}
        \begin{enumerate}
                \marking{1}{Showing that if $f$ is not constant, then $f(0)=0$ and $f(1)=1$}
                \marking{1}{Showing that if $f$ is not constant, then $f(n)=n$ for all integers $n \geq 2$} 
                \marking{1}{Showing that if $f$ is not constant, then $f(x) = x$ for all rational $x > 0$}
                \marking{2}{Showing that if $f$ is not constant, then $f(n) = -f(-n)$ for all integers 
                $n$\footnote{This point should also be awarded if the contestant finds a way to show 
                $f(n) = n$ for all negative integers $n$ without deriving $f(n)=-f(-n)$ explicitly.}}
                \marking{Sum of the corresponding}{Any linear combination of the above}
        \end{enumerate}
\end{enumerate}
